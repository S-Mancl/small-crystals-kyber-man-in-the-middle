\begin{chapter}[assets/background_negative]{statalelilla}{Special Slides}
\begin{itemize}
    \item Chapter slides
    \item Side-picture slides
\end{itemize}
\end{chapter}

%=======================================================================

\footlinecolor{statalelilla}
\begin{frame}[fragile]{Chapter slides}
\begin{itemize}
    \item Similar to \verb|frame|s, but with a few more options
    \item Opened with \verb|\begin{chapter}[<image>]{<color>}{<title>}|
    \item Image is optional, colour and title are mandatory
    \item There are seven ``official'' colours: \testcolor{maincolor}, \testcolor{stataledarkgreen}, \testcolor{statalegreen}, \testcolor{statalelightgreen}, \testcolor{statalered}, \testcolor{stataleyellow}, \testcolor{statalelilla}.
    \begin{itemize}
        \item Strangely enough, these are \emph{more} than the official colours for the footline.
        \item It may still be a nice touch to change the footline of following  slides to the same color of a chapter slide. Your choice.
    \end{itemize}
    \item Otherwise, \verb|chapter| behaves just like \verb|frame|.
\end{itemize}
\end{frame}

%=======================================================================

\begin{sidepic}{assets/background_alternative}{Side-Picture Slides}
\begin{itemize}
    \item Opened with \texttt{$\backslash$begin\{sidepic\}\{<image>\}\{<title>\}}
    \item Otherwise, \texttt{sidepic} works just like \texttt{frame}
\end{itemize}
\end{sidepic}

%=======================================================================

\begin{frame}
\frametitle{Fonts}
\begin{itemize}
    \item The paramount task of fonts is being readable
    \item There are good ones...
      \begin{itemize}
          \item {\textrm{Use serif fonts only with high-definition projectors}}
          \item {\textsf{Use sans-serif fonts otherwise (or if you simply prefer them)}}
      \end{itemize}
    \item ... and not so good ones:
    \begin{itemize}
        \item {\texttt{Never use monospace for normal text}}
        \item {\frakfamily Gothic, calligraphic or weird fonts: should always: be avoided}
\end{itemize}
\end{itemize}
\end{frame}

%=======================================================================

\begin{frame}[fragile]{Look}
\begin{itemize}
    \item To insert a final slide with the title and final thanks, use \verb|\backmatter|.
    \begin{itemize}
        \item The title also appears in footlines along with the author name, you can change this text with \verb|\footlinepayoff|
        \item You can remove the title from the final slide with \verb|\backmatter[notitle]|
    \end{itemize}
    \item The aspect ratio defaults to 16:9, and you should not change it to 4:3 for old projectors as it is inherently impossible to perfectly convert a 16:9 presentation to 4:3 one; spacings \emph{will} break
    \begin{itemize}    
        \item The \texttt{aspectratio} argument to the \texttt{beamer} class is overridden by the SINTEF theme
        \item If you \emph{really} know what you are doing, check the package code and look for the \texttt{geometry} class.
    \end{itemize}
\end{itemize}
\end{frame}

\footlinecolor{}